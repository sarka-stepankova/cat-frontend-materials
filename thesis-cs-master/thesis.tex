%%% Hlavní soubor. Zde se definují základní parametry a odkazuje se na ostatní části. %%%

% Meta-data o práci (je nutno upravit)
\input metadata.tex

% Vygenerujeme metadata ve formátu XMP pro použití balíčkem pdfx
\input xmp.tex

%% Verze pro jednostranný tisk:
% Okraje: levý 40mm, pravý 25mm, horní a dolní 25mm
% (ale pozor, LaTeX si k levému a hornímu sám přidává 1in=25.4mm)
\documentclass[12pt,a4paper]{report}
\setlength\textwidth{145mm}
\setlength\textheight{247mm}
\setlength\oddsidemargin{14.6mm}
\setlength\evensidemargin{14.6mm}
\setlength\topmargin{0mm}
\setlength\headsep{0mm}
\setlength\headheight{0mm}
% \openright zařídí, aby následující text začínal na pravé straně knihy
\let\openright=\clearpage

%% Pokud tiskneme oboustranně:
% \documentclass[12pt,a4paper,twoside,openright]{report}
% \setlength\textwidth{145mm}
% \setlength\textheight{247mm}
% \setlength\oddsidemargin{14.6mm}
% \setlength\evensidemargin{0mm}
% \setlength\topmargin{0mm}
% \setlength\headsep{0mm}
% \setlength\headheight{0mm}
% \let\openright=\cleardoublepage

%% Pokud práci odevzdáváme pouze elektronicky, vypadají lépe symetrické okraje
% \documentclass[12pt,a4paper]{report}
% \setlength\textwidth{145mm}
% \setlength\textheight{247mm}
% \setlength\oddsidemargin{7.1mm}
% \setlength\evensidemargin{7.1mm}
% \setlength\topmargin{0mm}
% \setlength\headsep{0mm}
% \setlength\headheight{0mm}
% \let\openright=\clearpage

%% Vytváříme PDF/A-2u
\usepackage[a-2u]{pdfx}

%% Přepneme na českou sazbu a fonty Latin Modern
\usepackage[czech]{babel}
\usepackage{lmodern}

% Pokud nepouživáme LuaTeX, je potřeba ještě nastavit kódování znaků
\usepackage{iftex}
\ifpdftex
\usepackage[utf8]{inputenc}
\usepackage[T1]{fontenc}
\usepackage{textcomp}
\fi

%%% Další užitečné balíčky (jsou součástí běžných distribucí LaTeXu)
\usepackage{amsmath}        % rozšíření pro sazbu matematiky
\usepackage{amsfonts}       % matematické fonty
\usepackage{amsthm}         % sazba vět, definic apod.
\usepackage{bm}             % tučné symboly (příkaz \bm)
\usepackage{booktabs}       % lepší vodorovné linky v tabulkách
\usepackage{caption}        % umožní definovat vlastní popisky plovoucích objektů
\usepackage{csquotes}       % uvozovky závislé na jazyku
\usepackage{dcolumn}        % vylepšené zarovnání sloupců tabulek
\usepackage{floatrow}       % umožní definovat vlastní typy plovoucích objektů
\usepackage{graphicx}       % vkládání obrázků
\usepackage{icomma}         % inteligetní čárka v matematickém módu
\usepackage{indentfirst}    % zavede odsazení 1. odstavce kapitoly
\usepackage[nopatch=item]{microtype}  % mikrotypografická rozšíření
\usepackage{paralist}       % lepší enumerate a itemize
\usepackage[nottoc]{tocbibind} % zajistí přidání seznamu literatury,
                            % obrázků a tabulek do obsahu
\usepackage{xcolor}         % barevná sazba

% Balíček hyperref, kterým jdou vyrábět klikací odkazy v PDF,
% ale hlavně ho používáme k uložení metadat do PDF (včetně obsahu).
% Většinu nastavítek přednastaví balíček pdfx.
\hypersetup{unicode}
\hypersetup{breaklinks=true}

% Balíčky pro sazbu informatických prací
\usepackage{algpseudocode}  % součást balíčku algorithmicx
\usepackage[Algoritmus]{algorithm}
\usepackage{fancyvrb}       % vylepšené prostředí verbatim
\usepackage{listings}       % zvýrazňování syntaxe zdrojových textů

% Cleveref může zjednodušit odkazování, ale jeho užitečnost pro češtinu
% je minimalní, protože nezvládá skloňování.
% \usepackage{cleveref}

% Formátování bibliografie (odkazů na literaturu)
% Detailní nastavení můžete upravit v souboru macros.tex.
%
% POZOR: Zvyklosti různých oborů a kateder se liší. Konzultujte se svým
% vedoucím, jaký formát citací je pro vaši práci vhodný!
%
% Základní formát podle normy ISO 690 s číslovanými odkazy
\usepackage[natbib,style=iso-numeric,sorting=none]{biblatex}
% ISO 690 s alfanumerickými odkazy (zkratky jmen autorů)
%\usepackage[natbib,style=iso-alphabetic]{biblatex}
% ISO 690 s citacemi tvaru Autor (rok)
%\usepackage[natbib,style=iso-authoryear]{biblatex}
%
% V některých oborech je běžnější obyčejný formát s číslovanými odkazy
% (sorting=none říká, že se bibliografie má řadit podle pořadí citací):
%\usepackage[natbib,style=numeric,sorting=none]{biblatex}
% Číslované odkazy, navíc se [1,2,3,4,5] komprimuje na [1-5]
%\usepackage[natbib,style=numeric-comp,sorting=none]{biblatex}
% Obyčejný formát s alfanumerickými odkazy:
%\usepackage[natbib,style=alphabetic]{biblatex}

% Z tohoto souboru se načítají položky bibliografie
\addbibresource{literatura.bib}

% Definice různých užitečných maker (viz popis uvnitř souboru)
\input macros.tex

%%% Titulní strana a různé povinné informační strany
\begin{document}
%%% Titulní strana práce a další povinné informační strany

%%% Nápisy na přední straně desek
%%% Pokud je práce ve slovenštině, desky mají být česky.

% Desky obvykle nesázíme, ale pokud je chcete přidat, změnte \iffalse na \iftrue
\iffalse

\pagestyle{empty}
\hypersetup{pageanchor=false}
\begin{center}

\large
Univerzita Karlova

\medskip

Matematicko-fyzikální fakulta

\vfill

{\huge\bf\ThesisTypeTitle}

\vfill

{\huge\bf\ThesisTitle\par}

\vfill
\vfill

\hbox to \hsize{\YearSubmitted\hfil \ThesisAuthor}

\end{center}

\newpage\openright
\setcounter{page}{1}

\fi

%%% Titulní strana práce
%%% Pokud je práce ve slovenštině, tato strana zůstává česky.

\pagestyle{empty}
\hypersetup{pageanchor=false}

\begin{center}

\centerline{\mbox{\includegraphics[width=166mm]{img/logo-cs.pdf}}}

\vspace{-8mm}
\vfill

{\bf\Large\ThesisTypeTitle}

\vfill

{\LARGE\ThesisAuthor}

\vspace{15mm}

{\LARGE\bfseries\ThesisTitle\par}

\vfill

\Department

\vfill

{
\centerline{\vbox{\halign{\hbox to 0.45\hsize{\hfil #}&\hskip 0.5em\parbox[t]{0.45\hsize}{\raggedright #}\cr
Vedoucí \ThesisTypeGenitive{} práce:&\Supervisor \cr
\ifx\ThesisType\TypeRig\else
\noalign{\vspace{2mm}}
Studijní program:&\StudyProgramme \cr
\fi
}}}}

\vfill

Praha \YearSubmitted

\end{center}

\newpage

%%% Strana s čestným prohlášením k práci
%%% Pokud je práce ve slovenštině, tato strana zůstává česky.

\openright
\hypersetup{pageanchor=true}
\vglue 0pt plus 1fill

\noindent
Prohlašuji, že jsem tuto \ThesisTypeAccusative{} práci vypracoval(a) samostatně a výhradně
s~použitím citovaných pramenů, literatury a dalších odborných zdrojů.
Beru na~vědomí, že se na moji práci vztahují práva a povinnosti vyplývající
ze zákona č. 121/2000 Sb., autorského zákona v~platném znění, zejména skutečnost,
že Univerzita Karlova má právo na~uzavření licenční smlouvy o~užití této
práce jako školního díla podle §60 odst. 1 autorského zákona.

\vspace{10mm}

\hbox{\hbox to 0.5\hsize{%
V \hbox to 6em{\dotfill} dne \hbox to 6em{\dotfill}
\hss}\hbox to 0.5\hsize{\dotfill\quad}}
\smallskip
\hbox{\hbox to 0.5\hsize{}\hbox to 0.5\hsize{\hfil Podpis autora\hfil}}

\vspace{20mm}
\newpage

%%% Poděkování

\openright

\noindent
\Dedication

\newpage

%%% Povinná informační strana práce

\openright
{\InfoPageFont

\vtop to 0.5\vsize{
\setlength\parindent{0mm}
\setlength\parskip{5mm}

Název práce:
\ThesisTitle

Autor:
\ThesisAuthor

\DeptType:
\Department

Vedoucí \ThesisTypeGenitive{} práce:
\Supervisor, \SupervisorsDepartment

Abstrakt:
\Abstract

Klíčová slova:
{\def\sep{\unskip, }\ThesisKeywords}

\vfil
}

\vtop to 0.49\vsize{
\setlength\parindent{0mm}
\setlength\parskip{5mm}

Title:
\ThesisTitleEN

Author:
\ThesisAuthor

\DeptTypeEN:
\DepartmentEN

Supervisor:
\Supervisor, \SupervisorsDepartmentEN

Abstract:
\AbstractEN

Keywords:
{\def\sep{\unskip, }\ThesisKeywordsEN}

\vfil
}

}

\newpage

%%% Další stránky budeme číslovat
\pagestyle{plain}


%%% Strana s automaticky generovaným obsahem práce

\tableofcontents

%%% Jednotlivé kapitoly práce jsou pro přehlednost uloženy v samostatných souborech
\chapter*{Úvod}
\addcontentsline{toc}{chapter}{Úvod}

V rámci GAČR projektu Unified Management of Multi-Model Data (č. 20-22276S) 
byla navržena sada nástrojů pro modelování a správu multi-modelových dat, 
např. MM-cat a MM-evocat. Nástroje vznikaly postupně a za podpory různých 
lidí. Díky tomu nemají jednotné rozhraní pro práci. Navíc jsou současná 
rozhraní často nepřehledná a uživatelsky nepřívětivá. Proto se v rámci 
ročníkového projektu zabýváme návrhem sjednoceného rozhraní. Důraz je 
kladen na návrh vhodného uživatelského rozhraní a jeho udržovatelnost.

Práce je soustředěna hlavně na nástroj MM-evocat. A to tak, aby bylo možné rozšířit aplikaci o další nástroje. Výsledkem práce je návrh low-fidelity prototypu MM-evocat a jeho otestování na vybraném vzorku reálných uživatelů.

\todo[inline, color=blue!30]{TODO: nahradit mezery za no-break space všude v dokumentu... pomocí \textasciitilde}


\chapter{Sada nástrojů}

\section{MM-evocat}

KOUPIL, Pavel; SVOBODA, Martin; HOLUBOVÁ, Irena. MM-cat: A tool for modeling and transformation of multi-model data using category theory. In: 2021 ACM/IEEE International Conference on Model Driven Engineering Languages and Systems Companion (MODELS-C). IEEE, 2021. p. 635-639.


\section{MM-infer}

KOUPIL, Pavel; HRICKO, Sebastián; HOLUBOVÁ, Irena. MM-infer: A Tool for Inference of Multi-Model Schemas. In: EDBT. 2022. p. 566-2.


\section{MM-evocat}

KOUPIL, Pavel; BÁRTÍK, Jáchym; HOLUBOVÁ, Irena. MM-evocat: A tool for modelling and evolution management of multi-model data. In: Proceedings of the 31st ACM International Conference on Information \& Knowledge Management. 2022. p. 4892-4896.


\section{MM-quecat}

KOUPIL, Pavel; CRHA, Daniel; HOLUBOVÁ, Irena. MM-quecat: A Tool for Unified Querying of Multi-Model Data. 2023.

\chapter{Cílové skupiny uživatelů}

Pro dobré nastavení projektu si definujeme cílové skupiny uživatelů, kteří budou naši aplikaci používat. Pokud už k~předchozím projektům, které se snažíme sjednotit, cílové skupiny existují, je dobré se s~nimi alespoň seznámit.

Jakmile nastavíme potenciální cílové skupiny uživatelů, zvolíme si konkrétní persony. Pod personou si můžeme představit detailní popis fiktivní osoby, která reprezentuje cílovou skupinu. Persony nám pomohou lépe rozebrat ekonomický status a~vlastnosti lidí ve skupinách. Nezapomínáme ani na jejich motivace a~zájmy. Nesnažíme se vystihnout všechny ze skupiny, ale vybíráme si hlavně stereotypní vlastnosti a~chování. Stejně jako kladné rysy a~zájmy chceme vystihnout i~co dotyčný nemá rád, případně vůči čemu je úplně odmítavý. 

Pro práci si definujeme jenom persony, které reprezentují uživatele na které cílíme. Lze ale i~definovat vylučující persony. Tedy takové, na které cílit nechceme.

Vytváření cílových uživatelů je důležitým nástrojem, který nás bude provádět v~dalších krocích projektu. Díky konkrétním představám nás budou lépe napadat konkrétní řešení a~realizace. Snažíme se uzpůsobit návrh personám a~vyvolat v~nich kladné emoce. Díky tomu získáme konkrétně zaměřený projekt, který nebude tolik odtržený od reálných uživatelů.

\section{Definice cílových skupin}

Protože k~předchozím projektům, které se snažíme sjednotit, persony a~skupiny neexistují, definovali jsme skupiny nové.

Nejdřív jsme potenciální uživatele rozdělili na začínající a~expertní. Jedni očekávají jednoduché koncepty, protože neznají ty složitější. Seznamují se s~doménou poprvé a~proto bývají mnohem rychleji frustrovaní, či odrazení. Nemají zkušenost ani nadhled, kterým by se učili aplikaci používat rychleji. Tak by se dala v~krátkosti vystihnout skupina začínajících. Druzí už mají nějakou zkušenost v~oboru, jsou zvyklí na dotazovací jazyky. Nerozumí přímo teorii co stojí za aplikací, ale spíše se opírají o~již zmíněnou zkušenost v~oboru.

Bylo těžké najít jednotné zástupce skupiny začínajících. I~proto jsme se rozhodli skupinu rozdělit na podskupinu mladých lidí a~dětí, kteří mají více sklony k~hravosti a~experimentování při objevování nových konceptů a~na podskupinu lidí ve věku 40 až 50 let, kteří se chtějí rekvalifikovat pro práci s~daty, ale zatím nemají vhodné zkušenosti.

Nakonec jsme si ještě pohrávali s~myšlenkou, že by aplikaci využíval stroj. Jednotlivé úkony by byly prováděny pomocí skriptů. Ačkoli se taková skupina nedá zařadit do skupin potenciálních uživatelů, je potřeba i~s~takovým využitím počítat. Nicméně ji dál nebudeme rozebírat.

\section{Popis cílových skupin}

S~charakterizací skupin nám pomůže představa konkrétních person. Do skupiny mladých začínajících můžeme zařadit bakalářského studenta informatického oboru. Pod starším začínajícím uživatelem si představíme padesátiletou pracovnici České pošty, která se chce rekvalifikovat pro práci s~daty. Člověk co vystudoval MFF UK a~dále pracuje v~oboru s~daty, třeba učitel Datového inženýrství, bude reprezentovat zástupce expertní skupiny uživatelů. 

Za pomocí person rozebereme jednotlivé skupiny potenciálních uživatelů. Každou skupinu krátce charakterizujeme, následně se snažíme vystihnout její chování. Nakonec se zamyslíme nad úkoly, které budou uživatelé v~aplikaci dělat.

\subsection{Skupina mladých lidí}

Je skupinou začínajících laiků, kteří se s~aplikací seznamují úplně poprvé. Nemají příliš zkušeností v~oboru, neznají dotazovací jazyk. Budou pracovat hlavně s~grafickým prostředím, které by jim mělo práci usnadnit a~motivovat je.

Chovají se spontánně, rozhodují se impulzivně a~experimentují. Z~toho co dělají vyzařuje hravost. Tímto přístupem velmi rychle prozkoumají různá zákoutí aplikace a~dobře otestují její funkčnost. Učí se velmi rychle, ale stejně tak je dokáže rychle odradit i~maličkost. Lépe si představují věci vizuálně a~ocení odezvu nebo odměnu za~úkol, který provedou. Odměna pro ně může být rychlá zpětná vazba, třeba odezva na úkony v~grafickém prostředí. Pro pohyb v~prostředí využívají hlavně počítačovou myš. Ocení barevnost, různorodost a~zajímavé podání.

V~aplikaci budou využívat možnosti dotazování nad daty, jednoduché modelování schematických kategorií, bez přidávání mapování a~dalších složitějších operací. Nebude se jich týkat převod konceptuálního znázornění dat do multimodelové databáze a~funkce nástroje MM-infer.

\subsection{Skupina začínajících ve věku 40-50 let}

Je druhá skupina začínajících uživatelů. Je to skupina, co se typicky chce rekvalifikovat do práce s~daty, ale mají zatím jen nevhodné zkušenosti. Jsou seznámeni se základní prací na počítači, umí používat webový prohlížeč, tabulkový a~textový procesor. Aplikace by pro ně měla být ideální na přeučení. Nemusí se učit komplexní technologie a~dál budou dělat úkony přes grafické rozhraní.

Narozdíl od mladých začínajících mají metodický přístup chování na webu, přistupují k~věci konzervativně. Upřednostňují umírněné podání grafiky. Jsou pro ně důležité popsané konkrétní kroky a~argumenty. Nepracují spontánně, dělají zadanou práci. Je potřeba servírovat jim množství práce postupně v~dávkách, aby nepřišlo informační zahlcení a~odrazení od aplikace. Stejně jako mladí začínající jsou klikací typ, k~práci využívají počítačovou myš. 

Úkony prováděné v~aplikaci jsou podobné skupině mladých začínajících lidí.

\subsection{Skupina expertních uživatelů}

Poslední skupina, která už zná koncepty ve světě databází. Zvyklá na textové dotazovací jazyky, databázové modely. Jsou schopní psát si vlastní skripty a~jinak automatizovat a~zefektivnit práci.

Nechovají se spontánně a~k~aplikaci přistupují hlavně metodicky. Dokážou se lépe orientovat v~aplikaci, díky zkušenosti z~oboru. Jsou spíše tolerantní, nemají přehnané nároky na aplikaci. Vědí, že je složité přecházet na jiný systém, narozdíl od začínajících uživatelů, které odradí i~maličkost. Více si potrpí na funkce, které jim urychlí práci. Myšleno možnost používat skripty a~klávesové zkratky. Ocení střízlivé podání a~strohá data. Pracuje jak s~grafickým rozhraním, tak i~textovým.

Expertní uživatel využije více funkcí aplikace. Není omezen jen na tvorbu schematické kategorie. Může přidávat mapování a~joby, převádět konceptuální znázornění dat do multimodelové databáze. Stejně tak bude využívat funkce nástroje MM-infer.

\chapter{HTA (Hierarchical Task Analysis)}

Na definování skupin uživatelů navážeme analýzou úkolů (Task Analysis) prováděných uživateli v~aplikaci. Hlavním cílem analýzy je dokumentovat a~strukturovat kroky nezbytné k~dokončení úkolů. Díky definování různých úkolů, rozdělíme komplexní úlohy na menší, lépe představitelné části. Vše zaznamenáme do hierarchické struktury, odtud se vzalo slovo Hierarchical. Soustředíme se jen na pozitivní průchod.

Hierarchickou analýzu úkolů můžeme rozdělit na několik částí. Jako první určíme úkoly, které chtějí uživatelé v~aplikaci dokončit. Každý hlavní úkol rozdělíme na podúkoly a~akce, které definují vztah mezi podúkoly. Navíc každému hlavnímu úkolu přiřadíme výchozí situaci. Následně z~jednotlivých komponent vytvoříme hierarchický diagram, v~našem případě se stromovou strukturou. Jako poslední analyzujeme posloupnost akcí a~snažíme se strukturu pochopit a~vylepšit.

\section{Vytváření diagramů}

V~charakterizaci cílových skupin (\ref{sec:popis-cilovych-skupin}) už jsme narazili na úkoly, které by uživatelé mohli dělat. Vybereme si z~nich tři, které budeme analyzovat. Začneme úkolem, kdy uživatel přidá databázovou komponentu. Další hlavní úkol je vytvoření jednoduchého schématu a~poslední jeho dekompozice. Vždy začínáme v~situaci systému otevřeného na hlavní stránce.

Podívejme se na první jednoduchý diagram (Obrázek \ref{obr03:hta1}). Uživatel přidá nový databázový systém k~již existujícím. Začíná ve spuštěném systému.

\begin{figure}[htb]
  \centering
  \includegraphics[height=70mm]{../img/HTA-1}
  \caption{Diagram pro přidání databázové komponenty.}
  \label{obr03:hta1}
\end{figure}

Kořen oznamuje jaký úkol bude uživatel provádět. V~první hladině máme kroky 1 až 4, které nám v~základních rysech ukazují, jak bude uživatel postupovat při plnění úkolu. Krok 3 dál rozdělíme na 3.1 Zadání dat a~3.2 Otestování. Na hraně pak definujeme postup uživatele při plnění kroků 3.1 a~3.2. Když není hrana popsaná, předpokládáme průběh kroků podle čísel v~hladině vzestupně. Konkrétní pořadí ale není podmíněné. Číslování 3 a~3.x vyjadřuje vztah mezi rodičem a~podúkoly.  Podtržené stavy v~HTA se dál nedělí. 

Takto vytvořený diagram nám neříká nic o~interakci uživatele s~konkrétním systémem. Díky tomu ale můžeme kroky rychle měnit a~upravovat. Lépe zhodnotíme, jestli uživatel nedělá příliš mnoho kroků pro dokončení úkolu, nebo naopak. Často se může stát, že jeden krok je ještě potřeba rozdělit.

Všechny tři analýzy úkolů, včetně přidání DB komponenty jsou k nahlédnutí v příloze \ref{priloha1}.

\chapter{Storyboardy}

Na vytvoření uživatelských skupin a~HTA diagramů navážeme tvořením Storyboardů. Storyboard je náčrt sekvence kroků, které uživatel v~aplikaci provádí. Každý důležitý bod je zaznamenán do buňky. Ty pak skládáme lineárně za sebe, až nám vznikne příběh připomínající komiks. Takové rozdělení umožňuje soustředit se na každý krok zvlášť. Zároveň se jedná o~další techniku, kde můžeme udělat rychlý náčrt a~podle potřeby upravovat. Jeden storyboard nezabere víc než jednu stránku.

Cílem storyboardu je odvyprávět příbeh o~použití aplikace konkrétním způsobem a~konkrétní personou. Jde nám hlavně o~vizuální zobrazení kroků, které jsme popisovali v~HTA diagramech. Zároveň nám může být návrh nápomocný, abychom si vizualizovali, jak budou uživatelé používat navrhovanou aplikaci a jak bude aplikace v náznacích vypadat.

V našem případě vezmeme reprezentanta každé skupiny uživatelů a~představíme si jeho chování pro jednu vybranou situaci z~námi vytvořeného HTA diagramu.

\section{Ukázka vytváření storyboardu}

Jako první začneme se zkušeným uživatelem. V aplikaci bude přidávat dvě různé databázové komponenty (PostgreSQL, MongoDB). Strukturu kroků zachováváme podobnou prvnímu hta (\ref{obr03:hta1}). Chceme zachytit uživatele, který se umí pohybovat v doméně, jenom nepoužíval náš nástroj.

Na Obrázku \ref{obr04:storyboard-1-text} můžeme vidět rychlý návrh obsahu storyboardu. Kroky v buňkách nejdřív místo kreslení popisujeme. Získáme tím přehled všech hlavních kroků, které bude uživatel dělat a ujasníme si, kolik asi bude potřeba buněk.

\begin{figure}[htb]
    \centering
    \includegraphics[height=150mm]{../img/storyboard-1-rychly-navrh}
    \caption{Rychlý návrh storyboardu pomocí textu v buňkách.}
    \label{obr04:storyboard-1-text}
\end{figure}

V rámci návrhu jsme vytvořili několik iterací. Některé snímky jsme z návrhu vyřadili, protože nebyli potřebné (např. červeně škrtlý snímek). První tři kroky jsme k původnímu návrhu přidávali jako širší úvod do situace.

Dáváme si pozor, abychom se v této fázi návrhu nenechali strhnout detaily, které zatím nejsou potřeba. O prvotních fázích návrhu se zmiňuje i kniha \cite{Refactoring_UI}. Navrhuje jednoduchý způsob, jak se nenechat pohltit detaily jako jsou typy písma, stíny, ikony atd. Podle autorů stačí vzít papír, tlustou fixu a tím si v podstatě znemožnit kreslení malých detailů. My jsme navíc omezení velikostí buněk, do kterých kreslíme.

Při vyjadřování situace pamatujeme na to, že aplikaci momentálně ovládá zkušený uživatel. V  aplikaci se zorientuje rychle a s úkolem nemá problém. Obrázek XXX je kompletní storyboard znázorněný pomocí kreseb místo textu.

I takhle porovnání mezi obrázky je vidět, že se lépe představuje ten druhý, díky vizualizaci... nějak odůvodnit

V návrhu je vidět... třeba to chování uživatele

\todo[inline, color=blue!30]{Tady ten konec ještě chci učesat. Přidat oříznutej obrázek storyboardu. Zároveň pak ještě přidat zbylé storyboardy do příloh.}



\chapter*{Závěr}
\addcontentsline{toc}{chapter}{Závěr}

\todo[inline, color=blue!30]{TODO: shrnout co se podařilo a co ne}

\todo[inline, color=blue!30]{TODO: Nezapomenout na konci nahradit mezery za no-break space všude v~dokumentu... pomocí \textasciitilde}

%%% Seznam použité literatury
%%% Seznam použité literatury (bibliografie)
%%%
%%% Pro vytváření bibliografie používáme biblatex. Ten zpracovává
%%% citace v textu (např. makro \cite{...}) a vyhledává k nim literaturu
%%% v souboru literatura.bib.
%%%
%%% Podívejte se na nastavení biblatexu v souboru thesis.tex.

%%% Vytvoření seznamu literatury. Pozor, pokud jste necitovali ani jednu
%%% položku, seznam se automaticky vynechá.

% Dovolíme položkám trochu vyčuhovat přes pravý okraj.
\def\bibfont{\hfuzz=2pt}

\printbibliography[heading=bibintoc,title=Literatura]

%%% Kdybyste chtěli bibliografii vytvářet ručně (bez biblatexu), lze to udělat
%%% následovně. V takovém případě se řiďte normou ISO 690 a zvyklostmi v oboru.

% \begin{thebibliography}{99}
%
% \bibitem{lamport94}
%   {\sc Lamport,} Leslie.
%   \emph{\LaTeX: A Document Preparation System}.
%   2. vydání.
%   Massachusetts: Addison Wesley, 1994.
%   ISBN 0-201-52983-1.
%
% \end{thebibliography}


%%% Obrázky v práci
%%% (pokud jich je malé množství, obvykle není třeba seznam uvádět)
\listoffigures

%%% Tabulky v práci (opět nemusí být nutné uvádět)
%%% U matematických prací může být lepší přemístit seznam tabulek na začátek práce.
\listoftables

%%% Použité zkratky v práci (opět nemusí být nutné uvádět)
%%% U matematických prací může být lepší přemístit seznam zkratek na začátek práce.
\chapwithtoc{Seznam použitých zkratek}

%%% Součástí doktorských prací musí být seznam vlastních publikací
\ifx\ThesisType\TypePhD
\chapwithtoc{Seznam publikací}
\fi

%%% Přílohy k práci, existují-li. Každá příloha musí být alespoň jednou
%%% odkazována z vlastního textu práce. Přílohy se číslují.
%%%
%%% Do tištěné verze se spíše hodí přílohy, které lze číst a prohlížet (dodatečné
%%% tabulky a grafy, různé textové doplňky, ukázky výstupů z počítačových programů,
%%% apod.). Do elektronické verze se hodí přílohy, které budou spíše používány
%%% v elektronické podobě než čteny (zdrojové kódy programů, datové soubory,
%%% interaktivní grafy apod.). Elektronické přílohy se nahrávají do SISu.
%%% Povolené formáty souborů specifikuje opatření rektora č. 72/2017.
%%% Výjimky schvaluje fakultní koordinátor pro zavěrečné práce.
\appendix
\chapter{Přílohy}

\section{První příloha}

\end{document}
