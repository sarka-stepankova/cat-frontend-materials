\chapter*{Závěr}
\addcontentsline{toc}{chapter}{Závěr}

\todo[inline, color=blue!30]{TODO: shrnout co se podařilo a co ne}

\todo[inline, color=blue!30]{TODO: Nezapomenout na konci nahradit mezery za no-break space všude v~dokumentu... pomocí \textasciitilde}

% v závěru vyloženě nepsat co se mi nepodařilo, napsat to co se mi podařilo a potom tam dám sekci do závěru: Future work - co chci udělat v budoucnosti (ne co se mi nepodařilo) -> ještě nejsou integrované další části a aktivně se vyvíjí a spolu s tím jak budou aplikace vyvíjeny, tak 
% a ještě zopakovat že pokud jsem chtěla něco zjistit, tak jsem zjistila (např na prototypech jsme si ověřili, že uživatelé…), to jsou ty doporučení pro implementaci
% něco se mi podařilo líp než jsem očekávala…
