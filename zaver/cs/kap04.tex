\chapter{Storyboardy}

\todo[inline, color=blue!30]{Psaní v~procesu, zatím jen nápady.}

Na vytvoření uživatelských skupin a~HTA diagramů navážeme tvořením Storyboardů. Vezmeme vždy reprezentanta každé skupiny uživatelů a~představíme si jeho chování pro jednu vybranou situaci z~diagramu HTA.


Storyboard je náčrt sekvence kroků, které uživatel v~aplikaci provádí. Každý důležitý bod je zaznamenán do buňky. Ty pak skládáme lineárně za sebe, až nám vznikne příběh připomínající komiks. Takové rozdělení umožňuje soustředit se na každý krok zvlášť. Zároveň se jedná o~další techniku, kde můžeme udělat rychlý náčrt a~podle potřeby upravovat. Jeden storyboard nezabere víc než jednu stránku.

Takové grafické uchopení nám může dát první náznaky toho, jak by mohla aplikace vypadat. Nesnažíme se jít do detailů. Jde nám hlavně o~vizuální zobrazení kroků, které jsme popisovali v~HTA diagramech.

Storyboardy jsou nápomocné pro návrháře, aby si vizualizovali, jak budou uživatelé používat navrhovanou aplikaci. 

Technika storyboardů nebyla původně určena pro návrh uživatelského rozhraní. -> Ozdrojovat

Detail comes later zmínit
