%%% Fiktivní kapitola s ukázkami citací

\chapter{Cílové skupiny uživatelů}

Pro dobré nastavení projektu si definujeme cílové skupiny, které budou naši aplikaci používat. Pokud už k předchozím projektům, které se snažíme sjednotit, cílové skupiny existují, je dobré se s nimi alespoň seznámit.

Jakmile nastavíme potenciální cílové skupiny uživatelů, zvolíme si konkrétní persony. Pod personou si můžeme představit detailní popis fiktivní osoby, která reprezentuje cílovou skupinu. Persony nám pomohou lépe rozebrat ekonomický status a vlastnosti lidí ve skupinách. Nezapomínáme ani na jejich motivace a zájmy. Nesnažíme se vystihnout všechny ze skupiny, ale vybíráme si hlavně stereotypní vlastnosti a chování. Stejně jako kladné rysy a zájmy chceme vystihnout i co dotyčný nemá rád, případně vůči čemu je úplně odmítavý. 

Pro práci si definujeme jenom persony, které reprezentují uživatele na které cílíme. Lze ale i definovat vylučující persony. Tedy takové, na které cílit nechceme.

Vytváření cílových uživatelů je důležitým nástrojem, který nás bude provádět v dalších krocích projektu. Díky konkrétním představám nás budou lépe napadat konkrétní řešení a realizace. Snažíme se uzpůsobit návrh personám a vyvolat v nich kladné emoce. Díky tomu získáme konkrétně zaměřený projekt, který nebude tolik odtržený od reálných uživatelů.

\section{Definice cílových skupin}

Protože k předchozím projektům, které se snažíme sjednotit, persony a skupiny neexistují, definovali jsme skupiny nové.

Nejdřív jsme potenciální uživatele rozdělili na začínající a expertní. Jedni očekávají jednoduché koncepty, protože neznají ty složitější. Seznamují se s doménou poprvé a proto bývají mnohem rychleji frustrovaní, či odrazení. Nemají zkušenost ani nadhled, kterým by se učili aplikaci používat rychleji. Tak by se dala v krátkosti vystihnout skupina začínajících. Druzí už mají nějakou zkušenost v oboru, jsou zvyklí na dotazovací jazyky. Nerozumí přímo teorii co stojí za aplikací, ale spíše se opírají o již zmíněnou zkušenost v oboru.

Bylo těžké najít jednotné zástupce skupiny začínajících. I proto jsme se rozhodli skupinu rozdělit na podskupinu mladých lidí a dětí, kteří mají více sklony k hravosti a experimentování při objevování nových konceptů a na podskupinu lidí ve věku 40 až 50 let, kteří se chtějí rekvalifikovat pro práci s daty, ale zatím nemají vhodné zkušenosti.

Nakonec jsme si ještě pohrávali s myšlenkou, že by aplikaci využíval stroj. Jednotlivé úkony by byly prováděny pomocí skriptů. Ačkoli se taková skupina nedá zařadit do skupin potenciálních uživatelů, je potřeba i s takovým využitím počítat. Nicméně ji dál nebudeme rozebírat.

\section{Popis cílových skupin}

S charakterizací skupin nám pomůže představa konkrétních person. Do skupiny mladých začínajících můžeme zařadit bakalářského studenta informatického oboru. Pod starším začínajícím uživatelem si představíme padesátiletou pracovnici České pošty, která se chce rekvalifikovat pro práci s daty. Člověk co vystudoval MFF UK a dále pracuje v oboru s daty, třeba učitel Datového inženýrství, bude reprezentovat zástupce expertní skupiny uživatelů. 

Za pomocí person rozebereme jednotlivé skupiny potenciálních uživatelů. Každou skupinu krátce charakterizujeme, následně se snažíme vystihnout její chování. Nakonec se zamyslíme nad úkoly, které budou uživatelé v aplikaci dělat.

\subsection{Skupina mladých lidí a dětí}

Je skupinou začínajících laiků, kteří se s aplikací seznamují úplně poprvé. Nemají příliš zkušeností v oboru, neznají dotazovací jazyk. Budou pracovat hlavně s grafickým prostředím, které by jim mělo práci usnadnit a motivovat je.

Chovají se spontánně, rozhodují se impulzivně a experimentují. Z toho co dělají vyzařuje hravost. Tímto přístupem velmi rychle prozkoumají různá zákoutí aplikace a dobře otestují její funkčnost. Učí se velmi rychle, ale stejně tak je dokáže rychle odradit i maličkost. Lépe si představují věci vizuálně a ocení odezvu nebo odměnu za úkol, který provedou. Odměna pro ně může být rychlá zpětná vazba, třeba odezva na úkony v grafickém prostředí. Pro pohyb v prostředí využívají hlavně počítačovou myš. Ocení barevnost, různorodost a zajímavé podání.

V aplikaci budou využívat možnosti dotazování nad daty, jednoduché modelování schematických kategorií, bez přidávání mapování a dalších složitějších operací. Nebude se jich týkat převod konceptuálního znázornění dat do multimodelové databáze a funkce nástroje MM-infer.

\subsection{Skupina začínajících ve věku 40-50 let}

Je druhá skupina začínajících uživatelů. Je to skupina, co se typicky chce rekvalifikovat do práce s daty, ale mají zatím jen nevhodné zkušenosti. Jsou seznámeni se základní prací na počítači, umí používat webový prohlížeč, tabulkový a textový procesor. Aplikace by pro ně měla být ideální na přeučení. Nemusí se učit komplexní technologie a dál budou dělat úkony přes grafické rozhraní.

Narozdíl od mladých začínajících mají metodický přístup chování na webu, přistupují k věci konzervativně. Upřednostňují umírněné podání grafiky. Jsou pro ně důležité popsané konkrétní kroky a argumenty. Nepracují spontánně, dělají zadanou práci. Je potřeba servírovat jim množství práce postupně v dávkách, aby nepřišlo informační zahlcení a odrazení od aplikace. Stejně jako mladí začínající jsou klikací typ, k práci využívají počítačovou myš. 

Úkony prováděné v aplikaci jsou podobné skupině mladých začínajících lidí.

\subsection{Skupina expertních uživatelů}

Poslední skupina, která už zná koncepty ve světě databází. Zvyklá na textové dotazovací jazyky, databázové modely. Jsou schopní psát si vlastní skripty a jinak automatizovat a zefektivnit práci.

Nechovají se spontánně a k aplikaci přistupují hlavně metodicky. Dokážou se lépe orientovat v aplikaci, díky zkušenosti z oboru. Jsou spíše tolerantní, nemají přehnané nároky na aplikaci. Vědí, že je složité přecházet na jiný systém, narozdíl od začínajících uživatelů, které odradí i maličkost. Více si potrpí na funkce, které jim urychlí práci. Myšleno možnost používat skripty a klávesové zkratky. Ocení střízlivé podání a strohá data. Pracuje jak s grafickým rozhraním, tak i textovým.

Expertní uživatel využije více funkcí aplikace. Není omezen jen na tvorbu schematické kategorie. Může přidávat mapování a joby, převádět konceptuální znázornění dat do multimodelové databáze. Stejně tak bude využívat funkce nástroje MM-infer.
