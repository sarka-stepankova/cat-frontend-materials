\chapter{Low-fidelity prototyp}

\todo[inline, color=blue!30]{UPDATE: Postupně přidávám další kapitoly. Není dokončené, zatím v procesu.}

\todo[inline, color=blue!30]{Do téhle kapitoly by se ještě mělo vejít porovnání různých nástrojů a proč jsme se tak rozhodli.}

Po storyboardech, které nám ukázali možné sekvence akcí, vytvoříme prototyp aplikace, se kterým budou interagovat uživatelé. V rámci ročníkového projektu vytvoříme jen low-fidelity prototyp. Tedy se budeme zabývat vytvořením rané verze aplikace, která nebude věrnou kopií aplikace výsledné. V této fázi, chceme hlavně vytvořit levně a rychle prototyp, na kterém můžeme vyhodnocovat nápady a zároveň ho můžeme otestovat na uživatelích spadajících do jednotlivých uživatelských skupin, které jsme si definovali.

\section{Paper prototyping}

Pro navrhování a testování uživatelského rozhraní, ve verzi low-fidelity, jsme si vybrali metodu prototypování na papír (paper prototyping). Celý proces od vytvoření prototypu až po uživatelské testování popisuje detailně kniha \cite{Paper_Prototyping}.

Začneme vytvořením papírových komponent aplikace (oken, menu, stránek, dialogových boxů, dat, pop-up zpráv). 
Po vytvoření prototypu provedeme usability otestování.
V takovém testu provádí uživatel realistické úkoly na papírovém prototypu. Uživatelé jsou vybraní podle uživatelských skupin, které jsme si definovali na začátku.
Testování pro nás znamená jedno sezení s uživatelem. Nejedná se o studii, kdy jsou prováděny série testů použitelnosti prováděných během několika dní.

\todo{Sem možná přidat ukázku (obrázek) nějakých komponent, jen pro představu}

Prototyp bude ovládaný osobou, která reprezentuje konání počítače, ale nevysvětluje, jak má rozhraní fungovat. Další zkušený člověk funguje jako zapisující pozorovatel. Pozoruje chování uživatele, zapisuje co uživatel dělá a co říká. Tímto způsobem provedeme velmi rychle iterativní testování na několika uživatelích. V průběhu můžeme aplikaci vylepšovat a všímat si opakujících se vzorů. Každé sezení s uživatelem končí vyplněním dotazníku spokojenosti s aplikací.

\section{Low-fidelity návrh}

Jak funguje návrh... V nezávislosti na existující aplikaci vytváříme.
Vytváříme jen pro počítač, ne pro telefon, pro tablet jen pro prohlížení.

Ukázat několik aplikací a porovnat, proč jsme se rozhodli tak a tak to udělat (jedna z nich https://app.sqldbm.com/PostgreSQL/DatabaseExplorer/Draft/). F-pattern, kam se dostanu po kolika kliknutích, 

\section{Uživatelské testování}

\todo{Do téhle podkapitoly dát obrázek z průběhu uživatelského testování}

Proč je dobré takový návrh vůbec testovat na uživatelích? (krom toho, že ho můžeme rychle měnit) str. 58 z \cite{Paper_Prototyping} No nitpicky (hnidopišský) feedback. Pokud se chceme vyhnout hnidopišské zpětné vazbě, může nám pomoct low-fidelity návrh. Komponenty nejsou jasně dané, uživatel se spíš zaměří na koncepty a funkčnost. Je totiž očividné, že jsme ještě nespecifikovali vzhled. Zároveň to povzbuzuje uživatele, aby nebyl pasivní, ale sám kreativně přemýšlel nad koncepty. Dokonce není neobvyklé měnit vzhled s uživatelem přímo při testování. Tím, že nedokončený návrh povzbuzuje ke kreativitě, se zabývali výzkumníci v článku \cite{Schumann_1996_AEN}.

Techniky pro pozorování uživatelů. Chceme se připravit, pro maximalizaci užitečnosti dat z otestování. Kroky pro pozorování uživatelů, tak abychom na konci věděli, kde mají uživatelé problém aplikaci používat. 
Připravit si úkoly, které budou uživatelé dělat. Mělo by se jednat o reálné úkoly, které budou uživatelé v aplikaci nejběžněji dělat. 
Dál najít uživatele co sedí na persony. Dávat si pozor, aby neznali aplikaci nebo naše názory na ni.
Pro testování vybrat místo, které je tiché a bez zbytečných vnějších vyrušování.
Popsat uživateli o co se jedná, že jsou zapojeni do raných fází návrhu. Zdůraznit, že testujeme aplikaci, ne uživatele.
Poprosit uživatele aby přemýšleli nahlas, aby říkali to co jim přijde na mysl, v průběhu plnění úkolů. Díky tomu prozkoumáme jejich očekávání od produktu, taky jejich úmysly a jejich strategie řešení problémů.
Upozornit, že nebudeme uživateli pomáhat při plnění úkolů. Je to nejlepší způsob jak zjistit jak uživatelé reálně interagují s aplikací. 
Po testování zodpovědět zbývající otázky od uživatele. Lze diskutovat, nějaké zajímavé chování, které uživatel měl při testování. Zeptat se na celkový dojem z aplikace. Tohle je popsané v knize \cite{Brenda_1990_art} (kapitola Some Techniques for Observing Users)

V dotazníku (a obecně) jenom otevřené otázky (abychom nikoho nenaváděli).


