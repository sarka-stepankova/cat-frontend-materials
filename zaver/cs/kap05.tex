\chapter{Paper prototyping}

\todo[inline, color=blue!30]{UPDATE: Zatím jen rozmýšlení co všechno v kapitole bude.}

\todo[inline, color=blue!30]{Do téhle kapitoly by se ještě mělo vejít porovnání různých nástrojů a proč jsme se tak rozhodli.}

V téhle knize se píše o paper prototyping: \href{https://books.google.cz/books?id=YgBojJsVLGMC&lpg=PP1&ots=1pVOrY-_2I&dq=Snyder%2C%20C.%20Paper%20prototyping.%20Morgan%20Kaufmann%2C%202003.&lr&hl=cs&pg=PA5#v=onepage&q=Snyder,%20C.%20Paper%20prototyping.%20Morgan%20Kaufmann,%202003.&f=false}(odkaz) (Snyder, C. Paper prototyping. Morgan Kaufmann, 2003.) ale my jsme tomu říkali hlavně paper mockup, jsou tam výhody a nevýhody toho použití

Po storyboardech, které nám ukázali možné sekvence akcí, vytvoříme prototyp, se kterým už budou interagovat uživatelé. V rámci ročníkového projektu vytvoříme jen low-fidelity prototyp. Tedy se budeme zabývat vytvořením rané verze aplikace, která nebude věrnou kopií aplikace výsledné. V této fázi, chceme hlavně vytvořit levně a rychle prototyp, na kterém můžeme vyhodnocovat nápady a zároveň ho můžeme otestovat na uživatelích spadajících do jednotlivých uživatelských skupin, které jsme si definovali.

Pro navrhování a testování uživatelského rozhraní, ve verzi low-fidelity, jsme si vybrali metodu prototypování na papír (paper prototyping). Od hlavy k patě popisuje kniha ...

Výsledné papírové prototypy... Jak funguje návrh...
Paper prototypes can be used for usability testing with real users. In such a test, the user performs realistic tasks by interacting with the paper prototype.

V nezávislosti na existující aplikaci vytváříme.

Technika používaná v raných fázích návrhu.

Není to comps (composition), kdy jenom . 

Testování v tom smyslu, že... Test refers to one session with a user. Study is a series of usability tests conducted over one to several days
