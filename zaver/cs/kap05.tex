\chapter{Low-fidelity prototyp}

\todo[inline, color=blue!30]{UPDATE: Postupně přidávám další kapitoly. Není dokončené, zatím v procesu.}

\todo[inline, color=blue!30]{Do téhle kapitoly by se ještě mělo vejít porovnání různých nástrojů a proč jsme se tak rozhodli.}

Po storyboardech, které nám ukázali možné sekvence akcí, vytvoříme prototyp aplikace, se kterým budou interagovat uživatelé. V rámci ročníkového projektu vytvoříme jen low-fidelity prototyp. Tedy se budeme zabývat vytvořením rané verze aplikace, která nebude věrnou kopií aplikace výsledné. V této fázi, chceme hlavně vytvořit levně a rychle prototyp, na kterém můžeme vyhodnocovat nápady a zároveň ho můžeme otestovat na uživatelích spadajících do jednotlivých uživatelských skupin, které jsme si definovali.

\section{Paper prototyping}

Pro navrhování a testování uživatelského rozhraní, ve verzi low-fidelity, jsme si vybrali metodu prototypování na papír (paper prototyping). Celý proces od vytvoření prototypu až po uživatelské testování popisuje detailně kniha \cite{Paper_Prototyping}.

Začneme vytvořením papírových komponent aplikace (oken, menu, stránek, dialogových boxů, dat, pop-up zpráv). 
Po vytvoření prototypu provedeme usability otestování.
V takovém testu provádí uživatel realistické úkoly na papírovém prototypu. Uživatelé jsou vybraní podle uživatelských skupin, které jsme si definovali na začátku.
Testování pro nás znamená jedno sezení s uživatelem. Nejedná se o studii, kdy jsou prováděny série testů použitelnosti prováděných během několika dní.

\todo{Sem možná přidat ukázku (obrázek) nějakých komponent, jen pro představu}

Prototyp bude ovládaný osobou, která reprezentuje konání počítače, ale nevysvětluje, jak má rozhraní fungovat. Další zkušený člověk funguje jako zapisující pozorovatel. Pozoruje chování uživatele, zapisuje co uživatel dělá a co říká. Tímto způsobem provedeme velmi rychle iterativní testování na několika uživatelích. V průběhu můžeme aplikaci vylepšovat a všímat si opakujících se vzorů. Každé sezení s uživatelem končí vyplněním dotazníku spokojenosti s aplikací.

\section{Low-fidelity návrh}

Jak funguje návrh... V nezávislosti na existující aplikaci vytváříme.
Vytváříme jen pro počítač, ne pro telefon, pro tablet jen pro prohlížení.

Ukázat několik aplikací a porovnat, proč jsme se rozhodli tak a tak to udělat. F-pattern


