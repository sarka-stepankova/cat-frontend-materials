\pagestyle{plain}  % kvuli cislovani stranek

\chapter{Sada nástrojů}


\section{MM-cat}

MM-cat framework je navržený tak, aby řešil složitosti spojené s~návrhem a~správou multi-modelových databází. Jeho hlavním úkolem je modelování multi-modelových schémat a~jejich mapování na příšlušné DBMS (Database Management System). Slouží i~jako základ pro rozšíření o~složitější úkoly.

Typickým scénářem použití je vytvoření ER (entity-relationship) schématu uživatelem. Takové schéma je pak automaticky převedeno do jednotné kategorické reprezentace, která umožňuje namapování na kombinaci DBMS. Díky specifikaci schématu je vytvořen skript s~příkazy CREATE, které se aplikují na přiřazené DBMS. Uživatel může následně dál upravovat ER~diagram, nebo provádět \mbox{SELECT} dotazy.

Na ukázce uživatelského rozhraní (Obrázek \ref{obr01:mm-cat}) je výsledek typického scénáře rozebraného v~předchozím odstavci. Na levé straně je uživatelem vytvořený ER diagram, na straně pravé pak jeho kategorická reprezentace. Na pravé straně ještě stojí za povšimnutí panel s~přístupovými cestami (značené oranžově) a~CREATE příkaz.

\begin{figure}[htb]
  \centering
  \includegraphics[height=75mm]{../img/mm-cat}
  \caption{Ukázka uživatelského rozhraní nástroje MM-cat.}
  \label{obr01:mm-cat}
\end{figure}

Další možnosti použití, i~podrobnější popis nástroje, lze nalézt v~článku \cite{MM_cat}.


\section{MM-infer}

Ne všechna data mají předem definované schéma. Nástroj MM-infer se snaží schéma zpětně zrekonstruovat z~již uložených multi-modelových dat. Je schopen odhalit intra- a~inter-modelové reference a~překrývání modelů. Zároveň nástroj umí efektivně zpracovávat velké množství dat.

MM-infer podporuje tři druhy DBMS, vybrané tak, aby bylo pokryto co nejvíc funkcí takových systémů. PostreSQL byl vybraný jako zástupce schema-full, relační DBMS. Neo4j reprezentuje schema-less a~MongoDB obojí. 

Rozhraní aplikace nás provede hned několika kroky, jejichž výsledkem je globální schéma pro vybranou množinu DBMS. Uživatel musí nejdřív takové DBMS vybrat (Obrázek \ref{obr01:mm-infer-load-database}). V~levém panelu je osa ukazující již splněné a~následující kroky, pro lepší orientaci v~procesu.

\begin{figure}[htb]
  \centering
  \includegraphics[height=75mm]{../img/mm-infer-load-database}
  \caption{Načtení databází v~nástroji MM-infer.}
  \label{obr01:mm-infer-load-database}
\end{figure}

Podle vybraných databází a~dalších parametrů je vytvořen RSD (Record Schema Description), sjednocující reprezentace. Po automatickém načtení je pak předána ruka uživateli, který kontroluje navržené kandidáty. Výsledkem je vizualizace globálního schématu (Obrázek \ref{obr01:mm-infer-result}).

\begin{figure}[htb]
  \centering
  \includegraphics[height=75mm]{../img/mm-infer-result}
  \caption{Vizualizace globálního schématu v~nástroji MM-infer.}
  \label{obr01:mm-infer-result}
\end{figure}

Aplikace příjemně provádí uživatele všemi potřebnými kroky. Nemá zbytečně příliš různých grafických prvků, které by uživatele zahlcovaly. Stejně jako mm-cat využívá dvou oken pro diagramy vedle sebe.

Další informace o~návrhu a~možnostech použití aplikace lze nalézt v~\cite{MM_infer}.

\section{MM-evocat}

MM-evocat je nástroj pro modelování a~správu evoluce v~multi-modelových datech. Řeší případy, kdy se struktura dat v~čase mění tak, aby například vyhovovala novým uživatelským požadavkům. Takové změny jsou pak propagovány napříč definovanými modely a~datovými instancemi.

Umožňuje vytvářet kategorický model, umí ho dekomponovat a~umožňuje uživateli, aby si zvolil mapování na podporované DBMS. MM-cat podporuje stejné druhy systémů jako framework MM-infer (PostreSQL, Neo4j, MongoDB).

Kategorický model, mapování a~základy konceptuálního modelování popisuje článek \cite{MM_evocat}.

MM-evocat má grafické webové rozhraní.

\todo[inline, color=blue!30]{TODO: rozepsat rozhraní. }

\section{MM-quecat}

Více v~\cite{MM_quecat}.

\todo[inline, color=blue!30]{TODO: dopsat vše ke quecat. }

\section{Nápady k této sekci, dočasná kapitola}

The Importance of the User Interface (v knize Galitz, W. O. The Essential Guide to User Interface Design. Wiley, 2002., tahle kniha je uvedená jako zdroj předmětu https://dcgi.fel.cvut.cz/courses/nur) -> proč vlastně děláme lepší návrh, je tam i user memory a obecně jak pracuje uživatel.

Dost podobné Designing Interfaces (co doporučil Jáchym)

Napsat motivaci pro sjednocení všech nástrojů.
