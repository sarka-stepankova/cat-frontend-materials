\chapter{Low-fidelity prototyp}

Po storyboardech, které nám ukázali možné sekvence akcí, vytvoříme prototyp aplikace, se kterým budou interagovat uživatelé. V rámci ročníkového projektu vytvoříme jen low-fidelity prototyp. Tedy se budeme zabývat vytvořením rané verze aplikace, která nebude věrnou kopií aplikace výsledné. V této fázi, chceme hlavně rychle vytvořit prototyp, na kterém můžeme vyhodnocovat nápady a zároveň ho můžeme otestovat na uživatelích spadajících do jednotlivých uživatelských skupin, které jsme si definovali.

\section{Paper prototyping (mockups)}

Pro navrhování a testování uživatelského rozhraní ve verzi low-fidelity jsme si vybrali jednu z využívaných metod, prototypování na papír (paper prototyping). Někdy také nazývaná jako tvorba paper mockupů. Celý proces od vytvoření prototypu až po uživatelské testování popisuje detailně kniha Paper prototyping od Carolyn Snyder \cite{Paper_Prototyping}.

Začneme vytvořením papírových komponent aplikace (oken, menu, stránek, dialogových boxů, dat, pop-up zpráv). 
Po vytvoření prototypu provedeme usability otestování.
V takovém testování provádí uživatel, v jednom sezení, realistické úkoly na papírovém prototypu. Nejedná se o studii, kdy jsou prováděny série testů použitelnosti během několika dní. Uživatelé jsou vybraní podle uživatelských skupin, které jsme si definovali na začátku.

Prototyp bude ovládaný osobou, která reprezentuje konání počítače, ale nevysvětluje, jak má rozhraní fungovat. Další zkušený člověk funguje jako zapisující pozorovatel. Pozoruje chování uživatele, zapisuje co uživatel dělá a co říká. Tímto způsobem provedeme velmi rychle iterativní testování na několika uživatelích. V průběhu můžeme aplikaci vylepšovat a všímat si opakujících se vzorů. Každé sezení s uživatelem končí vyplněním dotazníku spokojenosti s aplikací.

\subsection{Vytváření prototypu}

Papírový prototyp vytvoříme v nezávislosti na existujících aplikacích popsaných v 1. kapitole. V tomto momentě je dobré si ujasnit, na jakém zařízení budeme aplikaci používat. Implicitně jsme od začátku předpokládali, že půjde o počítačové rozhraní. Mobilní aplikace se k němu nepřidá, alespoň ne ve verzi s možností úprav. Stejně tak aplikace pro tablety. Tahle zařízení jsou užší a menší. Mají tak méně prostoru pro obsah, navigaci a interakci. Použití modelovacích a složitějších funkčností by bylo často složité až nemožné.

Pro obecné rozložení komponent jsme vzali v úvahu F-Pattern. Je pojmenovaný podle tvaru, který tvoří pohyby očí při skenování a čtení aplikace, připomínající písmeno F (Obr. \ref{obr05:fpattern}). Nehodí se pro telefony, to nás ale neomezuje. Dnes spíš návrh aplikací tíhne k rozmanitosti, proto na něm aplikaci nestavíme, jenom ho bereme v potaz. Hlavním jazykem aplikace bude angličtina. Uživatelé jsou zvyklí číst zleva doprava, proto je pattern orientovaný stejným směrem.

\begin{figure}[htb]
    \centering
    \includegraphics[height=70mm]{../img/F-Pattern}
    \caption{F-Pattern layout.}
    \label{obr05:fpattern}
  \end{figure}

Při návrhu se zaměřujeme na dostupnost různých částí aplikace. U těch, které jsou často používané se snažíme minimalizovat počet kliků, díky nimž se k hledané informaci dostaneme. Vezměme si jako příklad otevření formuláře pro přidání databázové instance. Na jeho otevření nám poslouží přesun jedním kliknutím do okna se správou databázových instancí a na něm tlačítko pro přidání, které je na stránce nejvýraznější.

Neoficiální pravidlo tří kliků (3-click rule) tvrdí, že přístup k jakékoli informaci v aplikaci by neměl být delší než tři kliknutí. Vzniklo z přesvědčení, že budou uživatelé frustrovaní a opustí aplikaci, když nenaleznou hledanou informaci do tří kliků. Pravidlo je ale příliš zjednodušené a nezohledňuje složité interakce. Dosažení nízkého počtu kliků s sebou nese riziko, že budeme ignorovat vše ostatní. Může vést i k frustraci uživatelů způsobené obšírnými navigacemi. Proto se pravidlem přímo neřídíme, ale snažíme se minimalizovat počet kliků, když to jde.

Chceme, aby se uživatel ocitl v pro něj známém (intuitivním) prostředí. Jako inspirace nám posloužily existující webové nástroje a desktopové aplikace. Úvodní obrazovka je tvořená z levé lišty (Obr. \ref{obr05:listy}), která obsahuje tlačítka pro otevření hlavních částí aplikace (spojení s databázovými instancemi, modelování, vytváření mapování), a volného prostoru. Podobnou lištu lze nalézt například v nástrojích pgModeler\footnote{\url{https://pgmodeler.io/}}, sqlDMB\footnote{\url{https://sqldbm.com/Home/}} a VS Code\footnote{\url{https://code.visualstudio.com/}} (Obr. \ref{obr05:listy}). Počítáme s tím, že až budou přibývat funkčnosti dalších nástrojů, budou přibývat tlačítka v levé liště.

\begin{figure}[htb]
  \centering
  \includegraphics[height=70mm]{../img/listy}
  \caption{Porovnání papírové lišty (vlevo) s lištami existujících aplikací.}
  \label{obr05:listy}
\end{figure}

V části aplikace pro správu databázových instancí vytvoříme lištu s již existujícími spojeními a jednoduchý formulář, který bude umožňovat přidání nové instance.

U vytváření části aplikace s modelováním schématu nám jako inspirace posloužil nástroj Draw.io\footnote{\url{https://app.diagrams.net/}}. Konkrétně hlavně u ER modelování lišta vlevo (Obr. \ref{obr05:er-lista}) a jednotlivé komponenty, které se přidávají na plochu. Do našeho papírového návrhu jsme totiž přidávali, místo vytváření cathegory schema, vytváření ER modelu. Proto, abychom uživateli představili něco jednoduššího s čím se už pravděpodobně setkal. 

\begin{figure}[]
  \centering
  \includegraphics[height=30mm]{../img/er-lista-drawio}
  \caption{ER lišta z nástroje Draw.io, sloužící jako inspirace.}
  \label{obr05:er-lista}
\end{figure}

Snažili jsme se ponechat co nejvíc volné bílé plochy pro pohodlné modelování a dost místa na umisťování komponent. Dodatečné lišty, sloužící k úpravě vlastnostní jednotlivých komponent, jsme vytvořili plovoucí, tak aby je uživatel mohl přetáhnout, pokud by někde překážely a aby zabíraly co nejméně místa.

Pro přidávání mapování do schématu jsme se drželi podobné struktury jako u vytváření schématu. Všechny papírové komponenty prototypu jsou naskenované v příloze \ref{priloha3}.

\section{Uživatelské testování}

Proč vůbec chceme low-fidelity návrh testovat na uživatelích? Kromě toho, že papírový návrh se dá rychle měnit, se můžeme vyhnout hnidopišské zpětné vazbě. O ní se zmiňují i v knize Paper Prototyping na straně 58 \cite{Paper_Prototyping}. Komponenty a barvy nejsou jasně dané, uživatel se spíš zaměří na koncepty a funkčnost. Je totiž očividné, že jsme ještě nespecifikovali vzhled. Schumann a další se v jednom ze svých článků \cite{Schumann_1996_AEN} zabývají tím, že nedokončený návrh povzbuzuje ke kreativitě a k tomu, aby uživatel nebyl pasivní a sám přemýšlel nad koncepty. 

Připravíme si úkoly, které budou uživatelé dělat. Budou to ty nejběžnější, které jsme rozebírali v předchozích kapitolách. Jmenovitě první úkol, kdy uživatel přidá databázovou komponentu. Další hlavní úkol je vytvoření jednoduchého schématu a~poslední jeho dekompozice.
Po připravení úkolů najdeme uživatele spadající do vytvořených uživatelských skupin. Dáváme si pozor, aby neznali aplikaci nebo naše názory na ni.

V kapitole Some Techniques for Observing Users knihy The Art of Human-computer Interface Design \cite{Brenda_1990_art} autorka popisuje techniky pro pozorování uživatelů, které se nám při testování budou hodit. Chceme zjistit, kde mají uživatelé problém aplikaci používat a co jim naopak vyhovuje.

Pro samotné testování vybereme místo, které je tiché a bez zbytečných vnějších vyrušování. Popíšeme uživateli o co se jedná a vyvětlíme, že jsou zapojeni do raných fází návrhu. Zdůrazníme, že testujeme aplikaci, ne uživatele. Poprosíme uživatele, aby přemýšleli nahlas a říkali to co jim přijde na mysl v průběhu plnění úkolů. Díky tomu prozkoumáme jejich očekávání od produktu, taky jejich úmysly a jejich strategie řešení problémů. Nakonec ještě upozorníme, že nebudeme uživateli pomáhat při plnění úkolů. Je to nejlepší způsob jak zjistit jak uživatelé reálně interagují s aplikací.

Po otestování zodpovíme zbývající otázky od uživatele. Případně diskutujeme nějaké zajímavé chování, které uživatel při testování měl. Závěrem pak bude vyplnění dotazníku uživatelem. Je důležité, aby v dotazníku byly jenom otevřené otázky, které uživatele nijak nenavádí. Ptáme se hlavně na pocity z aplikace, jestli něco neodvádělo pozornost uživatele a jestli dává smysl navigace po aplikaci.

\section{Závěr z uživatelského testování}

Papírový prototyp otestovalo sedm uživatelů pokrývajících všechny uživatelské skupiny. Zpětnou vazbu jsme získali ze zápisků od pozorovatele a z dotazníku. Všechny zápisky, dotazník i odpovědi na dotazník lze nalézt v příloze \ref{priloha4}

Pro představu průběhu uživatelského testování je na Obrázku \ref{obr05:ukazka-testovani} zachycena situace přidávání mapování do existujícího schématu.

\begin{figure}[htb]
  \centering
  \includegraphics[height=90mm]{../img/ukazka-testovani}
  \caption{Ukázka rozložení aplikace při uživatelském testování.}
  \label{obr05:ukazka-testovani}
\end{figure}

V průběhu testování na uživatelích jsme do aplikace přidali vyskakující okna s potvrzením o úspěšně provedené akci (například uložení souboru).

V dotazníku uživatelům připadala aplikace konzistentní a nic neodvádělo jejich pozornost od komponent na které v dalších krocích klikali. Díky konzistenci rozložení v aplikaci se uživatelé po splnění prvního úkolu pohybovali rychleji.

Uživatelé používali i otazník s nápovědou v hlavní liště. Původním předpokladem bylo, že uživatelé budou skeptiční a raději budou zkoušet klikat na různé prvky. Několik z nich si taky všimlo, že na úvodní obrazovku, která je tvořená jen z levé lišty a jinak je prázdná, by bylo dobré přidat nějaký úvodní text s nabídkou úkonů, které může uživatel začít dělat, pro rychlejší orientaci, když aplikaci otevře. Takovou nabídku jsme do aplikace nepřidávali, ale nakonec nás to napadlo a uživatelé nám domněnku potvrdili.

S jedním uživatelem jsme diskutovali rozložení hlavních funkcí. Poukázal na to, že by nebylo špatné schovat vytváření databázových spojení místo první funkce někam, aby k tomu měl přístup hlavně administrátor a expertní uživatel a obecně, aby se méně zkušeným uživatelům tahle možnost nenabízela. Určitě plánujeme zapojit tento poznatek do dalšího návrhu a v hlavní liště schovat nebo alespoň prohodit tlačítka určená hlavně pro expertní uživatele.

Obecně měli uživatelé problém s posledním úkolem, kde vytvářeli mapování. Protože je to ale úkol hlavně pro expertní uživatele, byly takové problémy očekávané. Naopak formulář pro přidávání DB komponenty pochopili všichni velmi rychle.
