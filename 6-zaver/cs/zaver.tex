\chapter*{Závěr}
\addcontentsline{toc}{chapter}{Závěr}

\todo[inline, color=blue!30]{TODO: shrnout co se podařilo a co ne}
% a opravit gramatiku
% Pavel Koupit dát jako konzultant

\todo[inline, color=blue!30]{TODO: Nezapomenout na konci nahradit mezery za no-break space všude v~dokumentu... pomocí \textasciitilde}

% v závěru vyloženě nepsat co se mi nepodařilo, napsat to co se mi podařilo a potom tam dám sekci do závěru: Future work - co chci udělat v budoucnosti (ne co se mi nepodařilo) -> ještě nejsou integrované další části a aktivně se vyvíjí a spolu s tím jak budou aplikace vyvíjeny, tak 
% a ještě zopakovat že pokud jsem chtěla něco zjistit, tak jsem zjistila (např na prototypech jsme si ověřili, že uživatelé…), to jsou ty doporučení pro implementaci
% něco se mi podařilo líp než jsem očekávala...

% tak na stránku a půlka má být future work
% chci jasně vypsat tohle jsme udělali, tohle taky, nebát se tam jasně se pochválit

% chci prostě říct co jsem dělala, trošku dýl, prostě jsem navrhla UI to je hlavní přínos a k tomu jsem udělala paper mockup a v rámci toho návrhu jsem nejdřív udělala tohle a tohle
% pak taky jako důležitý 
Tato práce se zaměřila na návrh uživatelského rozhraní pro framework MM-cat, který je určený pro správu multi-modelových databází. Hlavním přínosem je konzistentní návrh papírového prototypu, ve kterém se testovaní uživatelé poměrně rychle zorientovali. Ověřili jsme si, že uživatelé nebyli při plnění úkolů rozptylováni nadměrným množstvím tlačítek a funkcionalit. Pohybovali se v prostředí, které jim bylo známé díky inspiraci z rozložení již existujících nástrojů.

% jakou nám to přineslo výhodu, ověřili jsme si že něco platí pro nás
Postupný přístup od definování cílových skupin až po tvorbu storyboardů usnadnil tvorbu papírového prototypu. Testování na papírovém návrhu přineslo výhodu v podobě omezení drobných, detailních připomínek. Naopak se uživatelé zaměřovali na obecné rozložení a sami navrhovali užitečné úpravy.

% ještě víc rozvést, klidně jako podkapitola (future work), zase se rozepsat, to co jsem nestihla je v pořádku
Pro budoucí práci by bylo vhodné zaměřit se na přidání ostatních funkcionalit, abychom usnadnili jejich integraci s nástrojem MM-cat. Dále také vytvořit webové rozhraní na základě navrženého papírového prototypu a stejně jako v papírovém případě ho otestovat na uživatelích.




multimodal data, UI, UX, DBMS
